\documentclass{documento}

\entidade{Instituto de Matemática, Estatística e Computação Científica}
\data{Março de 2025}

\begin{document}

\noindent\textbf{Reitor: Carlos Henrique de Brito Cruz
Secretaria Geral:Patrícia Maria Morato Lopes Romano}

\noindent\textbf{\textit{Dispõe sobre o Regimento Interno da Congregação do Instituto de Matemática, Estatística e Computação Científica}}

\noindent O Reitor da Universidade Estadual de Campinas, na qualidade de Presidente do Conselho Universitário, tendo em vista o decidido pelo Conselho em sua 82ª Sessão Ordinária realizada em 27-5-03, baixa a seguinte deliberação:

\titulo{Da Organização da Congregação}

\capitulo{Da Composição}

\artigo A Composição da Congregação do Instituto de Matemática, Estatística e Computação Científica, com base no que dispõe o parágrafo único do Artigo 75 e do Artigo 76 e seus parágrafos, dos Estatutos e Artigos 137 e 138 do Regimento Geral da Universidade, fica assim constituída:

\inciso Oito dirigentes, a saber: Diretor, Diretor Associado, Coordenador de Ensino de Graduação, Coordenador do Curso de Pós-Graduação, Coordenador de Extensão, Chefe do Departamento de Matemática, Chefe do Departamento de Matemática Aplicada e Chefe do Departamento de Estatística;

\inciso Seis representantes das categorias docentes, a saber, dois MS-3, dois MS-5 e dois MS-6;

\inciso Representantes escolhidos por critérios da Unidade até o número de 10\% (dez por cento) do total de membros da Congregação que sejam docentes;

\inciso Dois representantes dos servidores Técnico-Administrativos;

\inciso Representação do Corpo Discente que terá número correspondente a 1/5 (um quinto) dos membros da Congregação. 

\paragrafounico O número de representantes a que se refere o Inciso V computa estudantes de graduação e pós-graduação.

\capitulo{Da Competência}

\artigo À Congregação, Órgão Superior do Instituto, compete, em concordância com o Artigo 143 do Regimento Geral: 

\inciso Pertinente à Legislação e Normas

\subinciso compor e encaminhar a lista tríplice para a escolha do Diretor, de acordo com os critérios e procedimentos estabelecidos no Regimento do Instituto, que contemplarão necessariamente o valor e o resultado da consulta à comunidade, realizada em conformidade com o Artigo 143 do Regimento Geral;
    
\subinciso elaborar o Regimento do Instituto e submetê-lo às instâncias superiores, após consulta prévia aos docentes, discentes e servidores do Instituto;

\subinciso elaborar o seu próprio Regimento;
    
\subinciso deliberar:

\subsubinciso sobre os regimentos internos dos Departamentos e do Conselho Interdepartamental;
        
\subsubinciso em caráter preliminar, sobre a criação, extinção ou fusão de Departamentos, centros ou quaisquer outras modificações na estrutura administrativa de ensino, de pesquisa e prestação de serviços do Instituto;
        
\subsubinciso em grau de recurso, nos casos previstos na legislação, sobre penalidade e sanções disciplinares;
    
\subinciso constituir comissões previstas no Regimento do Instituto e outras comissões de assessoramento;
    
\subinciso apreciar, em grau de recurso, decisões dos Departamentos e do Conselho Interdepartamental;
    
\subinciso resolver, em consonância com o ordenamento superior da Universidade, os casos omissos no Regimento do Instituto;
    
\subinciso manifestar-se, quando julgar oportuno, sobre quaisquer assuntos de interesse da Universidade.

\inciso Pertinente ao Corpo Docente

\subinciso propor:

\subsubinciso os quadros do Instituto ao Conselho Universitário, baseando-se nas propostas dos Departamentos;

\subsubinciso anualmente a atualização dos quadros de docentes do Instituto, baseando-se nas propostas dos Departamentos;

\subsubinciso a abertura de concursos para ingresso na carreira docente, baseando-se nas propostas dos Departamentos;

\subsubinciso ao Conselho Universitário a constituição de comissões julgadoras de concursos de Livre Docência, baseando-se nas propostas dos Departamentos;

\subinciso aprovar procedimentos internos de admissão, contratação, promoção, afastamento, licenças, demissão ou alteração de regime de trabalho de docentes, por solicitação dos órgãos competentes e em consonância com o ordenamento superior da Universidade;

\subinciso aprovar o relatório anual de atividades do Instituto.

\inciso Pertinente ao Orçamento

\subinciso definir critérios para a elaboração e execução do orçamento ordinário do Instituto, baseando-se nas propostas do Conselho Interdepartamental;

\subinciso deliberar:

\subsubinciso sobre o parecer do Conselho Interdepartamental emitido a respeito da proposta orçamentária ordinária do Instituto, a ser encaminhada às instâncias superiores da Universidade;

\subsubinciso sobre o relatório anual de execução do orçamento ordinário do Instituto, apresentado pela Diretoria.

\inciso Pertinente ao ensino, pesquisa e prestação de serviço

\subinciso aprovar as normas gerais e deliberar sobre as propostas dos Departamentos e coordenações de cursos, relativas a todos os cursos oferecidos pelo Instituto: currículos, programas, valor de créditos e pré-requisitos de disciplinas;

\subinciso opinar sobre as linhas de pesquisa estabelecidas no Instituto;

\subinciso definir: 

\subsubinciso critérios para o estabelecimento de convênios e contratos a serem executados pelo Instituto, baseando-se nas propostas do Conselho Interdepartamental relativos a convênios e contratos específicos, assim como sobre seus respectivos relatórios finais, à luz de política definida;

\subsubinciso critérios e estabelecer normas para a participação de docentes em atividades multidisciplinares que ultrapassem o âmbito do Instituto, baseando-se em propostas do Conselho;

\subinciso normatizar a prestação de serviços à comunidade, em consonância com o ordenamento superior da Universidade. 

\capitulo{Das Comissões}

\artigo A Congregação tem as seguintes comissões permanentes, de caráter consultivo e opinativo, cada uma delas definidas em regimento próprio, aprovado pela Congregação:

\inciso Comissão de Ensino de Graduação

\inciso Comissão de Pós-Graduação

\inciso Comissão de Extensão

\paragrafounico Às Comissões de Ensino de Graduação, Pós-Graduação e Extensão compete, nos termos de seus respectivos regimentos, assessorar a Diretoria, o Conselho Interdepartamental e, individualmente cada Chefe de Departamento.

\artigo A Congregação pode criar ou reativar comissões temporárias, de caráter consultivo e opinativo, destinadas a finalidades específicas indicadas pelo plenário, bem como pode alterar o tempo de atividade, as atribuições ou a composição de comissões temporárias previamente existentes.

\paragrafo As comissões poderão ser formadas por membros da Congregação ou convidados, devendo o relator ser necessariamente membro da Congregação.

\paragrafo A composição de cada comissão será decidida pelo plenário, tendo em vista as finalidades específicas a que ela se destina.

\artigo As comissões permanentes e temporárias só poderão funcionar com a presença da maioria de seus membros.

\paragrafo A presidência das comissões permanentes e temporárias caberá ao Diretor, quando presente em qualquer das reuniões.

\paragrafo O Diretor terá direito apenas ao voto de desempate.

\artigo As comissões permanentes e temporárias, no desempenho de suas atribuições, podem realizar as diligências que julgarem necessárias ao esclarecimento de aspectos das questões em exame.

\artigo A convite dos membros das comissões permanentes ou temporárias, podem participar de seus trabalhos, se houver consentimento da maioria dos membros e sem direito a voto, pessoas de reconhecida competência na matéria submetida à sua apreciação, ainda que não pertençam à Unicamp.

\paragrafounico As comissões podem estabelecer que a contribuição doconvidado seja feita por escrito.

\artigo Constituirá manifestação das comissões permanentes e temporárias o parecer aprovado pela maioria dos seus componentes.

\paragrafounico Os pareceres e os votos divergentes podem ser anexados à manifestação da comissão.

\capitulo{Dos membros}

\artigo Os membros componentes da Congregação e de suas comissões permanentes terão suplência em igual número, respeitada a ordem (i.e., 1º suplente, 2º...). O Diretor Associado será o suplente do Diretor. Cabe à Congregação indicar o suplente do Diretor Associado.

\artigo A freqüência às Sessões da Congregação é obrigatória, nos termos do Regimento Geral da Universidade.

\paragrafo Perderá o mandato o membro da Congregação ou de suas comissões permanentes que, sem causa justificada por escrito e aprovada pela Congregação, faltar, durante o ano civil, a duas sessões consecutivas ou três intercaladas, sejam elas sessões ordinárias ou extraordinárias. Assumirá o suplente até o término do mandato do titular. A justificativa deverá ser considerada na reunião subsequente.

\paragrafo No caso do membro ser um dirigente, nos termos do Artigo 1º, Inciso I, a perda de mandato será caracterizada tão somente pela perda do assento em plenário.

\paragrafo O suplente somente terá direito a voz e voto, durante a Sessão, quando tiver assinado lista de presença em substituição ao membro titular.

\artigo Nenhuma proposta de alteração ou modificação deste Regimento poderá ser aprovada sem o voto favorável da maioria absoluta dos membros da Congregação, em Sessão Ordinária realizada com quorum mínimo de dois terços (2/3), nem poderá ser aprovada sem que tenha sido apresentada em Sessão Ordinária anterior.

\paragrafounico O mesmo se aplica para o Regimento de qualquer outro órgão do Instituto, que tenha sido anteriormente aprovado pela Congregação.

\titulo{Do Funcionamento da Congregação}

\capitulo{Das Sessões}

\artigo A Congregação reunir-se-á, ordinariamente, a cada dois meses e, extraordinariamente, sempre que convocada pelo seu Presidente ou a requerimento de um terço (1/3) de seus membros ou durante uma reunião ordinária, quando aprovada pelo plenário.

\artigo As Sessões da Congregação são presididas pelo Diretor e secretariadas pelo Secretário da Congregação, que será um funcionário do Instituto designado pelo Diretor.

\paragrafo Em caso do impedimento do Diretor, a presidência será exercida pelo Diretor Associado e, na falta deste, sucessivamente pelo Chefe de Departamento com maior antigüidade no Instituto.

\paragrafo Em caso de empate na antigüidade de dois ou mais Chefes de Departamento, assumirá aquele que possuir maior nível na carreira e, persistindo o empate, aquele com maior titulação.

\artigo O Presidente detém o poder disciplinar das Sessões, que exercerá no interesse do bom andamento dos trabalhos e da preservação da ordem no plenário, respeitadas as atribuições da Congregação.

\artigo A pauta da Sessão deverá sempre que necessário estar acompanhada de pareceres e outros esclarecimentos necessários pertinentes e de cópias de atas de Sessões anteriores a serem submetidas para aprovação, quando os houver.

\paragrafo A documentação completa relativa à pauta da Sessão deverá ficar à disposição dos membros durante os dois dias antecedentes à Sessão.

\paragrafo As Sessões Ordinárias da Congregação serão realizadas em dependências do Instituto de Matemática, Estatística e Computação Científica, numa quinta-feira útil, às 13h30 min. 

\artigo As Sessões da Congregação são públicas

\paragrafo As pessoas assistentes que não sejam membros somente podem usar a palavra se e quando o Presidente ou o Plenário solicitar ou aquiescer.

\paragrafo O Presidente decidirá sobre a tramitação e a divulgação, parcial ou total, de assunto considerado sigiloso, podendo em conseqüência solicitar que as pessoas assistentes se retirem ou não compareçam.

\paragrafo Terão direito a usar a palavra pessoas em condições de prestar esclarecimentos sobre matéria técnica ou especializada constante do Expediente ou da Ordem do Dia, desde que presentes à Sessão por convite do Presidente ou por solicitação prévia de qualquer membro ao Presidente, que acolherá ou submeterá ao plenário.

\paragrafo O direito das pessoas convidadas de usar a palavra restringe-se ao assunto para o qual elas foram convidadas.

\paragrafo Todos os membros tem igual direito à voz.

\artigo Não havendo Sessão Ordinária ou Extraordinária por falta de número, poderá ser convocada nova Sessão mediante edital, observando o intervalo mínimo de quarenta e oito horas entre elas, desde que permaneça inalterada a pauta. No edital será especificado que não havendo quorum, conforme o Artigo 18, a Sessão se efetuará e deliberará com o número de membros presentes.

\paragrafounico A nova Sessão será convocada através de distribuição imediata de comunicado a todos os membros, efetuada pelo agente de convocação.

\artigo Ressalvando o disposto no caput do Artigo 17 e nos §§ 4º e 5º deste Artigo, a Congregação deliberará com a presença da maioria de seus membros. Para o início da Sessão haverá uma tolerância de quinze minutos.

\paragrafo Quando, no decurso de uma Sessão, se verificar que falta quorum para deliberar, será encerrada a Sessão, devendo a matéria não discutida ou votada ser apreciada, prioritariamente, na primeira Sessão que ocorrer, ordinária ou extraordinária.

\paragrafo Para fins exclusivos do disposto no Artigo 10, será considerado ausente da Sessão inteira o membro que, sem comunicação e justificativa por escrito à Presidência, apresentada antes do início da Sessão, retirar-se definitivamente antes de passadas duas (2) horas do horário de início, a não ser em caso de incidente imprevisível e de força maior, a critério do plenário.

\paragrafo Para fins de deliberação, serão considerados presentes os membros que, com a aquiescência da Presidência, retirarem-se temporariamente.

\paragrafo A critério do plenário, propostas específicas podem exigir a presença de dois terços (2/3) dos membros da Congregação para serem votadas e, uma vez votada esta exigência, ela não poderá ser relaxada.

\paragrafo Deliberações tomadas nas condições explicitadas no § 4º exigem o mesmo quorum citado neste parágrafo para serem alteradas.

\artigo Verificada a presença em número legal, o Presidente abrirá a Sessão, que se iniciará pela discussão e votação de atas de sessões anteriores, quando as houver.

\paragrafounico Sobre as atas, o membro falará o estritamente necessário sendo-lhe permitido, ainda, encaminhar à Presidência esclarecimento, indagação ou protesto por escrito.

\capitulo{Do Expediente}

\artigo O Expediente terá duração de até uma (1) hora, prorrogável por mais (30) minutos, a critério do plenário, e se destina ao trato de:

\inciso comunicações, explicações e relato de mensagens, ofícios, cartas, telegramas e similares, de interesse da Congregação recebidos ou encaminhados pela Presidência;

\inciso solicitações de licença à Sessão da Congregação e justificativas de faltas ou de saídas de membros antes do término da Sessão, recebidas pela Presidência;

\inciso propostas de moções ou de indicações da Congregação recebidas ou provenientes da Presidência;

\inciso apresentação de temas ou propostas para reflexão de matéria na Ordem do Dia da Sessão subsequente, ordinária ou extraordinária, recebidos ou provenientes da presidência;

\inciso manifestação ou pronunciamento dos membros inscritos para falar, após esgotada a apresentação pela Presidência, de assuntos enquadrados nos incisos I, II, III e IV.

\paragrafo Moções ou indicações da Congregação, bem como solicitações ou justificações incluídas nos Incisos II e IV e não acatadas pela Presidência, serão votadas imediatamente.

\paragrafo Inclusão de matéria na Ordem do Dia da Sessão em curso exige o voto favorável da maioria absoluta dos membros da Congregação, só podendo ser efetuada durante o Expediente.

\paragrafo Modificação na ordem dos itens da Ordem do Dia exige apenas votação simples, podendo ser efetuada durante o Expediente ou no decorrer do período da Sessão dedicado à própria Ordem do Dia.

\paragrafo Haverá sobre a mesa, uma lista para inscrição dos membros que quiserem usar a palavra durante o Expediente ou após a Ordem do Dia; a inscrição se fará junto ao Secretário da Congregação, devendo ser rigorosamente observada a ordem em que foi feita.

\paragrafo Se o membro não puder concluir sua exposição no Expediente, poderá fazê-lo depois de esgotada a pauta da Ordem do Dia.

\paragrafo Não se tratará, no Expediente, de nenhuma matéria constante da Ordem do dia.

\paragrafo Cabe ao Presidente, para preservar o tempo máximo dedicável ao Expediente, limitar, se necessário, o tempo disponível para cada orador.

\capitulo{Da Ordem do Dia}

\artigo Findo o Expediente, passar-se-á à Ordem do Dia.

\artigo As matérias serão incluídas na Ordem do Dia por determinação do Presidente, que harmonizará os critérios de antigüidade e importância.

\paragrafo Entende-se por matéria um determinado assunto ou processos da mesma natureza.

\paragrafo Quando a matéria compreender vários assuntos ou processos, cada um destes será considerado um item.

\paragrafo O Presidente poderá, a seu juízo ou por solicitação justificada de algum membro, designar uma das Comissões Permanentes da Congregação, um membro relator ou uma Comissão Especial de três membros, para estudar previamente e apresentar parecer sobre a matéria ou item constante na Ordem do Dia.

\artigo Os assuntos ou processos supervenientes à elaboração da pauta, e com caráter de urgência, poderão, a critério do Presidente ou por solicitação justificada de qualquer membro do Conselho Interdepartamental ou de todos os representantes de uma mesma categoria na Congregação, ou de três membros de categorias distintas, constar da Ordem do Dia Suplementar, e serão distribuídos aos membros com antecedência mínima de 24 horas.

\artigo Destaques serão aceitos até o início do Expediente. Qualquer membro poderá encaminhar à Diretoria, com antecedência os pedidos de destaque, para discussão e votação de determinada matéria ou item da Ordem do Dia.

\paragrafounico As matérias ou itens não destacados da Ordem do Dia serão considerados aprovados.

\artigo O Presidente ou qualquer membro, com a concordância do plenário decidida por maioria simples, poderá declarar prejudicada a matéria ou item de deliberação do plenário, retirando-a da pauta antes de concluída a discussão:

\inciso Por motivos justificados

\inciso Para reestudo ou instrução complementar

\paragrafounico O processo retirado de pauta nos termos do Artigo 26, deverá retornar à Congregação na reunião ordinária subsequente, ou antes, no caso da sua não inclusão na Ordem do Dia, justificada pelo Presidente, cabendo ao plenário decidir sobre a prorrogação do prazo.

\capitulo{Do Pedido de Vista}

\artigo O pedido de vista de matéria ou item constante da Ordem do Dia, apresentado por qualquer membro durante a Sessão, deverá ser sempre acompanhado de justificativa e será concedido, excetuando-se o disposto no § 3º deste mesmo Artigo.

\paragrafo A justificativa apresentada será transcrita na ata da Sessão em curso.

\paragrafo Os assuntos ou processos retirados da Ordem do Dia, em virtude de pedido de vista, serão devolvidos à Diretoria no prazo máximo de dez (10) dias corridos, a contar do recebimento da documentação pelo interessado, acompanhados obrigatoriamente de pronunciamento escrito, emitido pelo membro ou membros requerentes.

\paragrafo No caso da matéria ser julgada urgente pelo plenário, poderá o Presidente ou o plenário, prevalecendo este último, fixar prazo maior ou menor para a devolução, ou mesmo denegar o pedido de vista da matéria, fato que será anotado na ata da Sessão em que foi apresentado este pedido de vista.

\paragrafo Toda vez que ocorrer pedido de vista, o Presidente indagará ao plenário se mais algum membro também deseja solicitar vista do assunto ou processo em pauta.

\paragrafo Quando dois ou mais membros pedirem vista do mesmo assunto ou processo, o tempo concedido, nos termos dos §§ 2º e 3º, será entre eles dividido.

\paragrafo A inobservância de prazos implicará infração disciplinar e funcional, que será comunicada aos Órgãos e Autoridades Universitários superiores competentes, nos termos da legislação aplicável ao servidor público ou ao agente a ele equiparado. 

\paragrafo A Diretoria informará à Congregação sobre o não cumprimento dos prazos indicados, para os efeitos do § 6º.

\capitulo{Da Discussão e do Aparte}

\artigo Haverá sobre a mesa uma lista para inscrição dos membros que quiserem usar a palavra para discutir os assuntos constantes na Ordem do Dia, devendo ser rigorosamente observada a ordem de inscrição.

\artigo O tempo de cada orador poderá ser fixado pela Presidência, com a concordância do plenário, em função do número de oradores inscritos e da pauta da Sessão.

\paragrafounico O tempo máximo de um orador nunca excederá quinze (15) minutos, salvo anuência do plenário.

\artigo O aparte é a interrupção do orador para indagação ou esclarecimento relativo à matéria em discussão, e não ultrapassará um (1) minuto.

\paragrafo O membro só poderá apartear se houver solicitado o aparte ao orador, e este o houver permitido.

\paragrafo Não será permitido aparte:

\inciso paralelo a discurso ou como diálogo;

\inciso por ocasião de encaminhamento de votação;

\inciso quando o orador declarar, previamente, que não os concederá de modo geral;

\inciso até decisão sobre questão de ordem.

\paragrafo O tempo dedicado a apartes não será considerado disponível ao orador, embora o tempo dedicado à resposta dos apartes o seja.

\artigo A discussão de qualquer assunto, matéria ou item será encerrada pela presidência, com a aquiescência do plenário, passando-se, se for o caso, ao encaminhamento da votação.

\capitulo{Da Questão de Ordem}

\artigo Considera-se questão de ordem:

\inciso toda dúvida sobre a interpretação ou aplicação do Regimento Interno do Instituto ou do Regimento Geral da Universidade, na sua prática, ou sobre a inobservância de expressa disposição do Regimento Interno da Congregação;

\inciso questões relacionadas com o melhor andamento da Sessão.

\paragrafo As questões de ordem serão formuladas com clareza e com a indicação precisa das disposições que se pretende elucidar, ou cuja inobservância é patente, sob pena de o Presidente não permitir a continuação da sua formulação.

\paragrafo Durante a Ordem do Dia, somente podem ser formuladas questões de ordem ligadas à matéria que esteja sendo discutida ou votada.

\paragrafo Caberá ao Presidente resolver as questões de ordem ou delegar ao plenário a sua solução.

\capitulo{Do Encaminhamento da Votação}

\artigo Todas as propostas submetidas à apreciação da Congregação, deverão ser apresentadas por escrito.

\paragrafo Em qualquer momento da Ordem do Dia poderá ser apresentada uma proposta por um membro da Congregação, obedecida a ordem de inscrição.

\paragrafo Em qualquer momento, uma proposta poderá ser modificada ou retirada de pauta pelo membro da Congregação que a apresentou.

\artigo Encerrada a discussão e verificada a presença de quorum, ninguém poderá se retirar do recinto ou fazer uso da palavra, senão para encaminhar a votação e pelo prazo máximo de 02 (dois) minutos.

\paragrafo O encaminhamento da votação é medida preparatória desta e só se admitirá com relação a item ou matéria a ser votado e para fim de esclarecimento do plenário.

\paragrafo Serão feitos até dois encaminhamentos contra e dois a favor.

\artigo A matéria que abranger vários assuntos ou processos poderá ser votada em bloco, salvo destaques de determinado item.

\paragrafounico Se uma matéria comportar vários aspectos, o plenário poderá separá-los para discussão e votação.

\capitulo{Da Votação}

\artigo Só poderá ser votada matéria pertencente à Ordem do Dia.

\artigo Só se entrará em regime de votação quando o plenário se sentir suficientemente esclarecido sobre a matéria a ser votada.

\artigo Os processos de votação serão:

\inciso ativo, ou

\inciso nominal.

\artigo O processo comum de votação será o ativo, salvo dispositivo expresso, proposto por um membro da Congregação, aprovado pelo plenário.

\paragrafo Na votação ativa, o Presidente solicitará que se manifestem os membros da Congregação que forem na ordem, a favor, contra ou se abstiverem em relação à proposta. Em cada caso será feita a contagem de votos e o Presidente imediatamente proclamará o resultado final da votação.

\paragrafo Se algum membro da Congregação tiver dúvida quanto ao resultado proclamado, pedirá imediatamente verificação, que será realizada pelo processo nominal.

\paragrafo Será permitida a qualquer membro da Congregação, após a votação, fazer, sumariamente, declaração de voto, de duração de um minuto, ou entregá-la por escrito, durante a Sessão ao Secretário da Congregação, que dela dará conhecimento ao plenário.

\artigo O processo de votação nominal será utilizado quando disposições estatutárias ou regimentais assim o exigirem ou quando, sob proposta de um de seus membros, o plenário por ele optar. Nesse processo, os representantes responderão "sim", "não" ou "abstenção" à chamada nominal feita pelo Presidente. O Secretário anotará as respostas e proclamará o resultado final.

\artigo Será lícito ao membro da Congregação retificar o seu voto antes de proclamado o resultado da votação.

\artigo Qualquer membro da Congregação poderá apresentar seu voto por escrito, para constar da ata.

\artigo Ao Presidente cabe somente o voto de desempate.

\artigo As deliberações da Congregação corresponderão à vontade da maioria simples dos membros presentes no momento da votação, expressa através do resultado desta.

\paragrafo Se o número de abstenções for maior que o número de votos a favor e que o número de votos contra, considerados separadamente, o Presidente declarará a votação prejudicada e a proposta voltará a discussão.

\paragrafo A critério do plenário, a pedido de qualquer membro, decisões específicas podem exigir voto favorável da maioria absoluta dos membros presentes, ou da maioria absoluta dos membros da Congregação e, uma vez votada esta exigência, ela não poderá ser relaxada.

\capitulo{Da Ata da Sessão e do Encaminhamento das Deliberações}

\artigo O Secretário da Congregação lavrará ata da Sessão da qual constará:

\inciso A natureza da Sessão, o dia, a hora, o local de sua realização e o nome de quem a presidiu;

\inciso Os nomes dos membros presentes, bem como dos que não compareceram, consignando, a respeito destes, a circunstância de haverem ou não justificado sua ausência e o acatamento ou não desta justificação;

\inciso A discussão porventura havida a propósito de atas, a votação destas e, eventualmente, as retificações encaminhadas à mesa por escrito;

\inciso O Expediente;

\inciso As propostas e emendas apresentadas por escrito e as conclusões de pareceres;

\inciso A síntese dos debates e o resultado do julgamento de cada matéria ou item, com a respectiva votação;

\inciso A votação e as declarações de voto apresentadas por escrito.

\paragrafounico O registro em ata, na íntegra ou em resumo, de outras peças do autos ou de qualquer elemento além dos indicados, só se verificará quando encaminhados à mesa, por escrito, e mediante determinação do Presidente ou deliberação do plenário, prevalecendo esta última.

\artigo As decisões da Congregação que se refiram a casos de interesse individual serão comunicados por escrito aos interessados e, no caso de assuntos de interesse geral, a juízo do Presidente ou do plenário, prevalecendo este último, a Diretoria deverá tomar as providências cabíveis para sua divulgação.

\artigo Cabe à Diretoria encaminhar às instâncias competentes da Universidade deliberações da Congregação, que, por suas peculiaridades, exijam este encaminhamento para serem implementadas ou apreciadas.

\artigo As decisões da Congregação terão validade imediata após a sua aprovação, exceto nos casos onde houver exigência de aprovação da ata.

\titulo{Disposições Transitórias}

\artigo Dentro do prazo de 60 (sessenta) dias, a contar da vigência do presente Regimento, a Congregação deverá aprovar todas as normas e procedimentos para a plena vigência deste Regimento Interno.

\artigo Enquanto houver membro do corpo docente no nível funcional MS-2, ele poderá participar, para todos os efeitos, como candidato a representante, eleitor ou, caso eleito, como membro, na representação da categoria imediatamente superior (a de MS-3).

\artigo Esta deliberação entrará em vigor na data de sua publicação (Proc. 10-P-3730-03).

\end{document}